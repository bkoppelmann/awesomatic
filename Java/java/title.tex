\title{An Analyzer for Java}
\author{W. M. Waite \\ W. E. Munsil}
\maketitle

The purpose of this specification is to verify that the type analysis
modules of Eli 4.5 are useful for describing a modern object-oriented
language.
It implements the typing rules of Java 1.4, and accepts the entire
language.
Many programming errors that do not involve type analysis are not reported,
due to a lack of time and the particular goal of the project.

Complete name analysis for Java 1.4 requires a fixpoint algorithm for
resolving type inheritance, and this specification does not include that
algorithm.
As a result, some correct programs will be rejected with reports of
improper inheritance or undefined identifiers.
This deficiency is not relevant for the type analysis demonstration.

Chapter \ref{chap:lexical} defines the building blocks of Java text, and
Chapter \ref{chap:syntax} defines the programs that can be built from them.
Together, these definitions provide the \emph{phrase structure} of a Java
program.

Generally speaking, the phrase structure of a program is not a convenient
representation for analysis.
Chapter \ref{chap:tree} defines the \emph{abstract syntax tree} (AST),
a data structure that reflects the semantics of a Java program.
Eli deduces most of the transformations required to obtain the AST from the
phrase structure, but requires additional specification for some aspects
(Section \ref{sec:mapping}).

The computations specified in Chapters \ref{name} and \ref{type}
decorate the nodes of the AST with information about
the binding of identifiers and the type of expressions, respectively.
Chapter \ref{context} uses these decorations to report semantic errors.

We assume that Java packages are stored in a file system, and
Appendix \ref{file} defines how those packages are accessed.
All classes needed for a compilation are assumed to be available as source
text, and are added to the original text as individual compilation units.
The entire assemblage is then processed as a unit.

Appendix \ref{chap:debug} provides some additional information to aid in
interpreting definition table keys when examining attributes with Noosa.

The first version of this specification was developed in 1996 to test
whether Eli could deal with multiple-file input based on partial analysis
of the text.
That was successful, although the present specification uses a less
fine-grained mechanism.

In 1998 the specification was extended to the full Java 1.0 language and
used to implement some extensions for a Ph.D. thesis (Munsil, Wesley E.
``Intensive Inheritance with Applications to Java'', Department of Computer
Science, University of Colorado at Colorado Springs, 1998).

The Java 1.0 specification was enhanced to accept programs written in Java
1.4 in 2008.
All of the type analysis computations were rewritten to use Eli 4.5's type
analysis modules, and Java Names were disambiguated in the abstract syntax
tree by using a tree parser.

Eli 4.5 can generate an executable analyzer from the
specifications used to derive this document.
